\documentclass{report}
\usepackage[francais ]{babel}
\usepackage[utf8]{inputenc}
\usepackage[T1]{fontenc}
\usepackage{graphicx}
\newcommand{\projectName}{Mini Games}
\title{\projectName}

\author{M. \textsc{Friedli}, A. \textsc{Gillioz}, J. \textsc{Guerne}\\
He-Arc Ingénierie\\
2000 Neuchatel}
\date{\today{}}
\begin{document}
\maketitle{}
\chapter{Abstract}
La HES d'été permet aux étudiants de deuxième année d'étude dans le domaine de l'informatique
la possibilité de travailler sur un projet libre dans le but d'approfondir leurs connaissances.

Ce rapport décrit, explique les choix d'implémentations pris dans la réalisation de notre
projet \projectName.

Une planification des tâches ainsi qu'une spécification du travaille ont été réalisés dans le but
d'organiser au mieux le temps à disposition.

\tableofcontents

\chapter{Introduction}
\projectName  est une application offrant la possibilité de jouer à des minis jeux très classique tel
que le morpion ou la bataille navale en réseau et en multi-platforme. C'est à dire que deux personnes
l'une sur son téléphone android et l'autre sur son ordinateur auront la possibilité de se défier à une partie de jeu en ligne.
Deux grands outils on été utilisé pour faciliter l'implémentation de ce projet, il sagit du framework Libgdx, qui a servit à
deployer le même programme sur différentes platforme, et de la libraire kryonet, qui a elle facilitée les échanges réseau.

\chapter{Planification}

\chapter{Conventions}

\chapter{Identité graphique}


\chapter{LibGDX}
Dès le lancemment du projet, nous nous sommes orienté vers libGDX qui est un framework Java gratuit et open source permettant la conception de jeux vidéo.
Nous avons fait le choix de travailler avec ce framework en particulier, car nous avions découvert son existence quelque temps auparavant et, en apprenant à le connaître,
nous avons découvert à quel point il facilite le déploiement multi-platforme.
LibGDX nous a permis de gagner en temps précieux au niveau de l’implémentation puisqu’il propose nativement des fonctionnalités comme la gestion de stages ou de cameras qui sinon aurait dû être créés à la main.
Étant un framework connu et grandement utilisé, il est simple de trouver des renseignements ou de l’aide concernant sa façon de fonctionner. Nous n’avions aucune expérience dans son utilisation pourtant il ne nous a fallu que très peu de temps avant de commencer à d’obtenir de bons résultats.

\chapter{Kryonet}
Ayant travaillé cette année sur des échanges réseau en Java sans utiliser de libraire externe nous avons pu réaliser que la tâche était fastidieuse, c'est donc naturellement que nous
nous sommes tourné vers Kryonet qui est une librairie open source permettant de faciliter la communication entre différents clients. Un des points essentiel du projet était de pouvoir
garentir que LibGDX et Kryonet pouvaient cohabiter, après quelques tests et recherches nous avons pu réaliser que c'était bel et bien ée cas ( du moins pour le déploiment sur ordiateur et android).

\chapter{Architecture logicielle}

\chapter{Communication réseau}
Comme dit plus haut, le programme est une collection de minis jeux auxquels les clients auront la possibilités de jouer à
plusieurs. Le programme est décomposé en deux parties : la partie client et la partie serveur. La communication entre ces
deux parties est facilitée par l'utilisation de Kryonet une librairie Java open source conçu pour gérer les échanges réseau.

\section{Les packets}
Les données envoyées entre le client et le serveur transitent sous la forme de Packets, les packets sont des classes présentent
chez le client et chez le serveur enregistrer au lancement du service.

Pour un client, si une connexion à été établie, il lui est possible d'envoyer par TCP ou UDP des Packets au serveur. Dans le cas complémentaire,
pour recevoir les différents Packets émmanent du serveur le client met en place un listener qui va automatiquement gérer la réception des Packets et analyser leur contenu.
L'implémentaion du serveur est identique à la nuance près qu'elle laisse le choit de la personne (de l'adresse) à qui sera envoyé le Packets. En effet toutes les infos, tous les Packets,
transitent par le serveur, c'est lui ensuite qui les traitent et les renvoient aux différents clients conscernés.

\section{Initialisation de la connexion}
A l'ouverture du programme le client est invité à entrer une adresse de serveur et un pseudo pour tenter ensuite de se connecter. Afin de faciliter l'entrée de l'adresse du serveur la fonction de
découverte des hôtes fournie par Kryonet a été utilisée, elle va fournir une liste d'adresse qui seront ensuite présentées à l'utilisateur sous la forme d'une liste déroulante
si l'utilisateur sélectionne un élément de la liste l'adresse est automatiquement copié dans le champs de l'adresse du serveur.

\section{Communication en jeu}

\chapter{Minis jeux}
Dans ce chapitre seront présentés les différents minis jeux mis en place dans ce projet aisni que leur implémentation.

Le serveur garde un historique des jeux sous la forme d'une liste d'objets contenant les informations sur les différents joueurs opposés durant la partie.

\section{Morpion}
Le morpion est un jeu très simple dans lequel deux joueurs s'affronte sur un plateau de 3 x 3 cases. Chaque joueur possède des caractère (joueur 1 'x' et joueur 2 'o')
tour à tour ils vont devoir placer ces cartacères dans le plateau de jeu dans le but de faire une ligne horizontale,verticale ou encore une diagonale.
Si la partie se finit sans qu'aucun joueur n'est réussit à remplir une ligne/ une diagonale c'est un match nul.

\section{Bataille navale}


\chapter{Conclusion}

\section{Problematique}

\section{Améliorations}

\chapter{Bibliographie}

\end{document}
